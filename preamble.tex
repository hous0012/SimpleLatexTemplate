% The preamble.tex contains all packages, defines custom commands and environments, and holds various configurations
% Add or remove packages according to the needs of your specific document.
% ------------------------------------------------------------------------------

% Base Configuration -----------------------------------------------------------
\usepackage[T1]{fontenc}	% Use 8-bit T1 fonts for better language support
\usepackage[utf8]{inputenc}	% Allow utf-8 input for international characters
\usepackage[round]{natbib}	% Bibliography management with round parentheses
\usepackage[colorlinks=true, citecolor=blue, urlcolor=blue, linkcolor=black]{hyperref}	% Clickable hyperlinks with custom colors

% Math Packages -----------------------------------------------------------------
\usepackage{mathtools}		% Extended math commands from AMS
\usepackage{amssymb}		% Additional math symbols
\usepackage{amsthm}		% Theorem environments
\usepackage{bm}			% Bold math symbols via \bm command

% Styling and Formatting --------------------------------------------------------
\usepackage{enumitem}		% Enhanced list environments (itemize, enumerate, description)
\usepackage{tocbibind}		% Include bibliography/index/contents in Table of Contents

% Optional: Modify margins, spacing, and headers/footers
% Uncomment to use, and adjust values as needed
\usepackage[margin=1in]{geometry}	% Set page margins
%\usepackage{parskip}			% Add space between paragraphs, remove indentations
%\usepackage{setspace}			% Line spacing
%\usepackage{fancyhdr}			% Custom headers and footers

% Figures and Tables ------------------------------------------------------------
\usepackage{graphicx}		% Enhanced support for graphics
\usepackage{subcaption}		% Support for sub-figures and sub-captions
\usepackage{booktabs}		% Publication quality tables
\usepackage{multirow}		% Support for multi-row cells in tables

% Optional: Advanced tables and figures
% Uncomment to use, and adjust as needed
%\usepackage{tabularx}		% Advanced table features
%\usepackage{longtable}		% Multi-page tables

% Algorithms --------------------------------------------------------------------
\usepackage{algorithm}		% Float wrapper for algorithms
\usepackage{algorithmic}	% Algorithm typesetting environment

% Code Listings (optional) ------------------------------------------------------
% Uncomment to use the listings package for code snippets
%\usepackage{listings}		% Source code listings
%\lstset{
	%    basicstyle=\ttfamily\small,
	%    frame=single,
	%    breaklines=true,
	%    captionpos=b,
	%}

% Glossaries and Acronyms (optional) --------------------------------------------
% Uncomment to use, adjust settings as needed
%\usepackage[acronym]{glossaries}  % Create glossaries and lists of acronyms
%\makeglossaries

% Cross-referencing -------------------------------------------------------------
% Uncomment to enhance cross-referencing capabilities
%\usepackage[nameinlink]{cleveref} % Intelligent cross-referencing

% Font Customization (optional) -------------------------------------------------
% Uncomment to switch to Times New Roman, or customize as needed
%\usepackage{times}                % Times New Roman font

% Custom Theorem Environments ---------------------------------------------------
\theoremstyle{plain}			% Bold title, italic body text
\newtheorem{thm}{Theorem}		% Theorems
\newtheorem{prop}[thm]{Proposition}	% Propositions, numbered like theorems
\newtheorem{lem}[thm]{Lemma}		% Lemmas, numbered like theorems
\newtheorem{cor}[thm]{Corollary}	% Corollaries, numbered like theorems

\theoremstyle{definition}		% Bold title, upright body text
\newtheorem{defn}[thm]{Definition}	% Definitions, numbered like theorems
\newtheorem{remark}{Remark}		% Remarks, separate numbering
\newtheorem{exmp}[thm]{Example}		% Examples, numbered like theorems

% Custom Commands and Shortcuts -------------------------------------------------
% Custom font size for headings
% Usage example: \section[TOC Title]{\size{15}{Displayed Title}}
\newcommand{\size}[2]{{\fontsize{#1}{0}\selectfont#2}}

% Custom wide bar for math symbols
\makeatletter
\newcommand*\rel@kern[1]{\kern#1\dimexpr\macc@kerna}
\newcommand*\widebar[1]{%
	\begingroup
	\def\mathaccent##1##2{%
		\rel@kern{0.8}%
		\overline{\rel@kern{-0.8}\macc@nucleus\rel@kern{0.2}}%
		\rel@kern{-0.2}%
	}%
	\macc@depth\@ne
	\let\math@bgroup\@empty \let\math@egroup\macc@set@skewchar
	\mathsurround\z@ \frozen@everymath{\mathgroup\macc@group\relax}%
	\macc@set@skewchar\relax
	\let\mathaccentV\macc@nested@a
	\macc@nested@a\relax111{#1}%
	\endgroup
}
\makeatother

% Fix for spacing around adjusted parentheses
\let\originalleft\left
\let\originalright\right
\renewcommand{\left}{\mathopen{}\mathclose\bgroup\originalleft}
\renewcommand{\right}{\aftergroup\egroup\originalright}

% Hide TOC's heading
\makeatletter
\renewcommand\tableofcontents{%
	\@starttoc{toc}%
}
\makeatother

% Mathematical shortcuts
\DeclareMathOperator*{\argmin}{arg\,min}
\DeclareMathOperator*{\argmax}{arg\,max}
\DeclareMathOperator{\st}{subject\,\,to}

\def\R{\mathbb{R}}
\def\C{\mathbb{C}}
\def\N{\mathbb{N}}
\def\Z{\mathbb{Z}}
\def\E{\mathbb{E}}
\def\P{\mathbb{P}}

% Miscellaneous Configurations
\renewcommand\qedsymbol{$\blacksquare$}		% Custom QED symbol
\renewcommand{\bibname}{References}		% Rename bibliography
\captionsetup{font=it,labelfont=bf}		% Format figure captions

% -----------------------------------------------------------------------------------
