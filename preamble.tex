%%% The preamble.tex contains all packages, defines custom commands and environments, and holds various configurations

%% Packages
%% -----------------------------------------------------------------
% Base packages
\usepackage[T1]{fontenc}            % use 8-bit T1 fonts
\usepackage[utf8]{inputenc}         % allow utf-8 input
\usepackage[round]{natbib}          % bibliography package
\usepackage[colorlinks=true,citecolor=blue,urlcolor=blue,linkcolor=black]{hyperref}               % insert clickable hyperlinks
\usepackage{xpatch}                 % macro patching commands

% Math packages
\usepackage{mathtools}              % math commands and environments
\usepackage{amssymb}                % additional math symbols
\usepackage{amsthm}                 % theorem environments
\usepackage{bm}                     % improved bold faces for math symbols
\usepackage{mathrsfs}               % mathscr symbols

% Figure packages
\usepackage{graphicx}               % handling of images
\usepackage{booktabs}               % improved tables
\usepackage{listings}               % source code listings
\usepackage{tikz}                   % drawing vector graphics

\usepackage{caption}                % custimisation of captions
\usepackage{subcaption}             % for grouping figures together

% Styling packages
\usepackage{microtype}              % microtyphography
\usepackage{emptypage}              % supresses page numbering on empty pages
\usepackage[margin=3cm]{geometry}   % adjust page margins
\usepackage{enumitem}               % better list environments
\usepackage{cleveref}               % enchances LATEX's cross-referencing features
\usepackage{tocbibind}              % include bibliography in table of contents

\usepackage{fancyhdr}               % easy customisation of headers and footers
\usepackage{titlesec}               % easy customisation of section titles

% Font packages, Times New Roman
\ifdefined\TimesFont
\usepackage{times}
\fi

% Paragraph skip
\ifdefined\ParSkip
\usepackage{parskip}
\fi
%% -----------------------------------------------------------------

%% Custom commands and environments
%% -----------------------------------------------------------------
% Theorem environments
\theoremstyle{plain}                 % bold title, italic body text
\newtheorem{thm}{Theorem}            % theorems
\newtheorem{prop}[thm]{Proposition}  % propositions, numbered like theorems
\newtheorem{lem}[thm]{Lemma}         % lemmas, numbered like theorems
\newtheorem{cor}[thm]{Corollary}     % corollaries, numbered like theorems

\theoremstyle{definition}            % bold title, upright body text
\newtheorem{defn}[thm]{Definition}   % definitions, numbered like theorems
\newtheorem{remark}{Remark}          % remarks, seperate numbering
\newtheorem{exmp}[thm]{Example}      % examples, numbered like theorems

% Miscellaneous
\newcommand{\size}[2]{{\fontsize{#1}{0}\selectfont#2}}      % custom fontsize of headings; e.g. usage, \section[TOC Title]{\size{15}{Diplayed Title}}
\renewcommand\qedsymbol{$\blacksquare$}                     % tombstone symbol as q.e.d

% Fix for spacing around adjusted parentheses
\let\originalleft\left
\let\originalright\right
\renewcommand{\left}{\mathopen{}\mathclose\bgroup\originalleft}
\renewcommand{\right}{\aftergroup\egroup\originalright}

% Widebar
\makeatletter
\newcommand*\rel@kern[1]{\kern#1\dimexpr\macc@kerna}
\newcommand*\widebar[1]{%
  \begingroup
  \def\mathaccent##1##2{%
    \rel@kern{0.8}%
    \overline{\rel@kern{-0.8}\macc@nucleus\rel@kern{0.2}}%
    \rel@kern{-0.2}%
  }%
  \macc@depth\@ne
  \let\math@bgroup\@empty \let\math@egroup\macc@set@skewchar
  \mathsurround\z@ \frozen@everymath{\mathgroup\macc@group\relax}%
  \macc@set@skewchar\relax
  \let\mathaccentV\macc@nested@a
  \macc@nested@a\relax111{#1}%
  \endgroup
}
\makeatother

% Shortcuts
\DeclareMathOperator*{\argmin}{arg\,min}
\DeclareMathOperator*{\argmax}{arg\,max}
\DeclareMathOperator{\st}{subject\,\,to}

\def\R{\mathbb{R}}
\def\C{\mathbb{C}}
\def\N{\mathbb{N}}
\def\Z{\mathbb{Z}}
\def\E{\mathbb{E}}
\def\P{\mathbb{P}}
%% -----------------------------------------------------------------

%% Various configurations
%% -----------------------------------------------------------------
% Hide TOC's heading
\makeatletter
\renewcommand\tableofcontents{%
    \@starttoc{toc}%
}
\makeatother

% Rename 'Bibliography' to 'References'
\renewcommand{\bibname}{References}

% Format figure captions
% http://mirrors.dotsrc.org/ctan/macros/latex/contrib/caption/caption-eng.pdf
\captionsetup{font=it,labelfont=bf}

% Source code style
% https://en.wikibooks.org/wiki/LaTeX/Source_Code_Listings
\lstdefinestyle{custompy}{
  belowcaptionskip=\baselineskip,
  breaklines=true,
  language=R,
  %numbers=left,
  showstringspaces=false,
  basicstyle=\footnotesize\ttfamily,
  keywordstyle=\color{black},
  %backgroundcolor=\color{gray!10},
  commentstyle=\color{black},
  stringstyle=\color{black},
  otherkeywords={},
  morekeywords={},
  deletekeywords={}
}
\lstset{language=R,style=custompy,captionpos=b}
\renewcommand{\lstlistingname}{Listing}
\renewcommand{\lstlistlistingname}{List of Listings}

% Autoref settings
% https://en.wikibooks.org/wiki/LaTeX/Labels_and_Cross-referencing#autoref
\def\partautorefname{Part}
\def\chapterautorefname{Chapter}
\def\sectionautorefname{Section}
\def\subsectionautorefname{Subsection}
\def\figureautorefname{Figure}
\def\tableautorefname{Table}
\def\lstinputlistingautorefname{Listing}
%% -----------------------------------------------------------------